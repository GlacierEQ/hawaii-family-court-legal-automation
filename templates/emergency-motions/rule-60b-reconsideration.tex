\documentclass[12pt,letterpaper]{article}
\usepackage[margin=1in]{geometry}
\usepackage{setspace}
\usepackage{titlesec}
\usepackage{fancyhdr}
\usepackage{lastpage}

% Hawaii Family Court formatting requirements
\doublespacing
\pagestyle{fancy}
\fancyhf{}
\rhead{Page \thepage\ of \pageref{LastPage}}
\renewcommand{\headrulewidth}{0pt}

% Title formatting for Hawaii Family Court
\titleformat{\section}
  {\normalfont\fontsize{12}{15}\bfseries}
  {}{0em}{}
\titleformat{\subsection}
  {\normalfont\fontsize{12}{15}\bfseries}
  {}{0em}{}

\begin{document}

% Hawaii Family Court Caption
\begin{center}
{\fontsize{12}{15}\selectfont
\textbf{IN THE FAMILY COURT OF THE FIRST CIRCUIT}\\
\textbf{STATE OF HAWAII}\\
\vspace{0.5in}
\textbf{CASEY DEL CARPIO BARTON,}\\
\vspace{0.1in}
Plaintiff,\\
\vspace{0.1in}
v.\\
\vspace{0.1in}
\textbf{TERESA DEL CARPIO BARTON,}\\
\vspace{0.1in}
Defendant.\\
\vspace{0.2in}
}
\end{center}

% Case information box
\noindent\begin{tabular}{|p{2in}|p{4in}|}
\hline
\textbf{CASE NO.} & \textbf{1FDV-23-0001009} \\
\hline
\textbf{FILING PARTY} & Casey Del Carpio Barton, Pro Se \\
\hline
\textbf{ADDRESS} & 2665 Liliha St Apt A, Honolulu, HI 96817 \\
\hline
\textbf{PHONE} & (808) 936-5654 \\
\hline
\textbf{EMAIL} & glacier.equilibrium@gmail.com \\
\hline
\end{tabular}

\vspace{0.5in}

\begin{center}
{\fontsize{14}{17}\selectfont\textbf{EMERGENCY MOTION FOR RELIEF FROM JUDGMENT}}\\
{\fontsize{14}{17}\selectfont\textbf{PURSUANT TO HFCR RULE 60(b)}}\\
{\fontsize{14}{17}\selectfont\textbf{CHILD WELFARE PROTECTION AND DUE PROCESS RESTORATION}}
\end{center}

\vspace{0.3in}

\section{INTRODUCTION AND RELIEF REQUESTED}

COMES NOW, Plaintiff Casey Del Carpio Barton (\textbf{"Casey"}), proceeding pro se, and respectfully moves this Honorable Court pursuant to Hawaii Family Court Rules (HFCR) Rule 60(b) for relief from the default judgment entered against him on [DATE]. This emergency motion is filed to protect the welfare of the minor child, Kekoa Del Carpio Barton (\textbf{"Kekoa"}), age [AGE], who is currently experiencing documented mental health deterioration under the existing custody arrangement.

\subsection{Specific Relief Requested}

Plaintiff respectfully requests that this Honorable Court:

\begin{enumerate}
\item VACATE the default judgment entered against Plaintiff;
\item SET ASIDE all orders entered as a result of the default judgment;
\item RESTORE Plaintiff's right to participate meaningfully in these proceedings;
\item SCHEDULE an emergency hearing to address the minor child's deteriorating mental health;
\item MODIFY the custody arrangement to protect Kekoa's welfare and mental health;
\item RESTORE meaningful father-child contact prior to the upcoming November birthdays;
\item GRANT such other relief as the Court deems just and proper.
\end{enumerate}

\section{LEGAL STANDARD FOR RULE 60(b) RELIEF}

HFCR Rule 60(b) provides that the court may relieve a party from a final judgment for the following reasons:

\begin{enumerate}
\item Mistake, inadvertence, surprise, or excusable neglect;
\item Newly discovered evidence that by due diligence could not have been discovered in time;
\item Fraud, misrepresentation, or other misconduct by an adverse party;
\item The judgment is void;
\item The judgment has been satisfied, released, or discharged;
\item Any other reason justifying relief from the operation of the judgment.
\end{enumerate}

\textbf{Legal Authority:} \textit{[Case citations to be inserted based on research]}. The Hawaii Supreme Court has consistently held that Rule 60(b) motions should be liberally granted when justice requires, particularly in family law matters involving children's welfare.

\section{FACTUAL BACKGROUND AND PROCEDURAL HISTORY}

\subsection{Case Timeline and Default Judgment}

[AUTOMATED TIMELINE INSERTION POINT]

The default judgment was entered under circumstances that demonstrate excusable neglect and newly discovered evidence regarding Kekoa's deteriorating mental health condition.

\subsection{Child Welfare Crisis - Newly Discovered Evidence}

\textbf{Critical Mental Health Deterioration:}

Since the entry of default judgment and the resulting custody arrangement, Kekoa has been diagnosed with clinical depression. This constitutes newly discovered evidence that could not have been available at the time of the original proceedings, as the mental health deterioration is a direct result of the separation from his father and the current custody environment.

\textbf{Documented Neglect Patterns:}

\begin{enumerate}
\item \textbf{Personal Care Neglect}: Evidence shows inconsistent bathing and hygiene maintenance;
\item \textbf{Supervision Failures}: Kekoa sustained a broken arm under Teresa's care, with evidence suggesting inadequate supervision;
\item \textbf{Emotional Neglect}: Pattern of using iPad as substitute for meaningful parental interaction;
\item \textbf{Mental Health Impact}: Progressive psychological deterioration since father-child separation.
\end{enumerate}

\subsection{Critical Timeline - November Birthdays}

The timing of this motion is critical given the approaching November birthdays:

\begin{itemize}
\item \textbf{November 17, 2025}: Casey's birthday - symbolic of family unity
\item \textbf{November 29, 2025}: Kekoa's birthday - crucial for child's emotional stability
\end{itemize}

The prolonged separation between father and child during these significant dates would constitute additional psychological harm to an already vulnerable child experiencing clinical depression.

\section{GROUNDS FOR RELIEF UNDER RULE 60(b)}

\subsection{Excusable Neglect - Rule 60(b)(1)}

[EVIDENCE INSERTION POINT - Procedural circumstances leading to default]

The failure to appear at the settlement conference resulted from [SPECIFIC CIRCUMSTANCES] which constitute excusable neglect under Hawaii law.

\subsection{Newly Discovered Evidence - Rule 60(b)(2)}

\textbf{Kekoa's Clinical Depression Diagnosis:}

The minor child's diagnosis of clinical depression constitutes newly discovered evidence that:

\begin{enumerate}
\item Could not have been discovered through due diligence at the time of original proceedings;
\item Materially affects the child's best interests analysis;
\item Demonstrates the harmful impact of the current custody arrangement;
\item Requires immediate judicial intervention to protect the child's welfare.
\end{enumerate}

\subsection{Any Other Reason Justifying Relief - Rule 60(b)(6)}

\textbf{Child Welfare Emergency:}

The deteriorating mental health of the minor child presents extraordinary circumstances that justify relief under Rule 60(b)(6). The Hawaii Supreme Court has consistently held that the protection of children's welfare constitutes grounds for relief even when other specific subsections may not strictly apply.

\section{CHILD'S BEST INTERESTS REQUIRE IMMEDIATE ACTION}

\subsection{Mental Health Crisis}

Kekoa's clinical depression diagnosis represents a mental health emergency that requires immediate parental intervention and support. The current custody arrangement, which severely limits meaningful father-child contact, exacerbates the child's psychological distress and prevents access to the emotional support and stability that Casey's involvement would provide.

\subsection{Father-Child Bond Restoration}

The evidence demonstrates a strong historical bond between Casey and Kekoa that has been disrupted by the current custody arrangement. Restoration of meaningful contact is essential for:

\begin{enumerate}
\item Addressing Kekoa's mental health needs through paternal support;
\item Providing emotional stability during critical developmental years;
\item Ensuring access to both parents' love, guidance, and support;
\item Preventing further psychological harm from prolonged separation.
\end{enumerate}

\section{CONSTITUTIONAL DUE PROCESS CONCERNS}

The entry of default judgment without adequate notice and opportunity to be heard raises significant constitutional due process concerns. The fundamental liberty interest in the parent-child relationship requires rigorous procedural protections, particularly in emergency proceedings affecting child custody.

\section{CONCLUSION AND PRAYER FOR RELIEF}

For the foregoing reasons, Plaintiff respectfully requests that this Honorable Court grant this Emergency Motion for Relief from Judgment and:

\begin{enumerate}
\item VACATE the default judgment entered against Plaintiff;
\item SET ASIDE all custody and financial orders entered as a result of default;
\item SCHEDULE an emergency hearing within ten (10) days;
\item MODIFY custody arrangements to protect Kekoa's mental health;
\item RESTORE meaningful father-child contact immediately;
\item ENSURE Kekoa receives appropriate mental health support;
\item GRANT such other relief as justice requires.
\end{enumerate}

\vspace{0.3in}

\textbf{Time is of the essence.} Kekoa's mental health is deteriorating, and the approaching November birthdays represent a critical opportunity for family healing and reconnection that must not be lost.

\vspace{0.5in}

Respectfully submitted,

\vspace{1in}

\rule{3in}{0.4pt}\\
Casey Del Carpio Barton\\
Plaintiff, Pro Se\\
2665 Liliha St Apt A\\
Honolulu, HI 96817\\
(808) 936-5654\\
glacier.equilibrium@gmail.com

\vspace{0.3in}

\section{CERTIFICATE OF SERVICE}

I hereby certify that a true and correct copy of the foregoing Emergency Motion for Relief from Judgment was served upon all parties as follows:

\vspace{0.2in}

\textbf{Teresa Del Carpio Barton}\\
Defendant\\
[ADDRESS]\\
Service Method: [METHOD]\\
Date: [DATE]

\vspace{0.2in}

\textbf{Scot Stuart Brower, Esq.}\\
Attorney for Defendant\\
[ADDRESS]\\
Service Method: [METHOD]\\
Date: [DATE]

\vspace{0.5in}

\rule{3in}{0.4pt}\\
Casey Del Carpio Barton\\
Plaintiff, Pro Se\\
Date: \_\_\_\_\_\_\_\_\_\_\_\_\_\_\_\_

\end{document}
